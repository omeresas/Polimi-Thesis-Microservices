% A LaTeX template for MSc Thesis submissions to 
% Politecnico di Milano (PoliMi) - School of Industrial and Information Engineering
%
% S. Bonetti, A. Gruttadauria, G. Mescolini, A. Zingaro
% e-mail: template-tesi-ingind@polimi.it
%
% Last Revision: October 2021
%
% Copyright 2021 Politecnico di Milano, Italy. NC-BY

\documentclass{Configuration_Files/PoliMi3i_thesis}

%------------------------------------------------------------------------------
%	REQUIRED PACKAGES AND  CONFIGURATIONS
%------------------------------------------------------------------------------

% CONFIGURATIONS
\usepackage{parskip} % For paragraph layout
\usepackage{setspace} % For using single or double spacing
\usepackage{emptypage} % To insert empty pages
\usepackage{multicol} % To write in multiple columns (executive summary)
\setlength\columnsep{15pt} % Column separation in executive summary
\setlength\parindent{0pt} % Indentation
\raggedbottom  

% PACKAGES FOR TITLES
\usepackage{titlesec}
% \titlespacing{\section}{left spacing}{before spacing}{after spacing}
\titlespacing{\section}{0pt}{3.3ex}{2ex}
\titlespacing{\subsection}{0pt}{3.3ex}{1.65ex}
\titlespacing{\subsubsection}{0pt}{3.3ex}{1ex}
\usepackage{color}

% PACKAGES FOR LANGUAGE AND FONT
\usepackage[english]{babel} % The document is in English  
\usepackage[utf8]{inputenc} % UTF8 encoding
\usepackage[T1]{fontenc} % Font encoding
\usepackage[11pt]{moresize} % Big fonts

% PACKAGES FOR IMAGES
\usepackage{graphicx}
\usepackage{transparent} % Enables transparent images
\usepackage{eso-pic} % For the background picture on the title page
\usepackage{subfig} % Numbered and caption subfigures using \subfloat.
\usepackage{tikz} % A package for high-quality hand-made figures.
\usetikzlibrary{}
\graphicspath{{./Images/}} % Directory of the images
\usepackage{caption} % Coloured captions
\usepackage{xcolor} % Coloured captions
\usepackage{amsthm,thmtools,xcolor} % Coloured "Theorem"
\usepackage{float}

% STANDARD MATH PACKAGES
\usepackage{amsmath}
\usepackage{amsthm}
\usepackage{amssymb}
\usepackage{amsfonts}
\usepackage{bm}
\usepackage[overload]{empheq} % For braced-style systems of equations.
\usepackage{fix-cm} % To override original LaTeX restrictions on sizes

% PACKAGES FOR TABLES
\usepackage{tabularx}
\usepackage{longtable} % Tables that can span several pages
\usepackage{colortbl}

% PACKAGES FOR ALGORITHMS (PSEUDO-CODE)
\usepackage{algorithm}
\usepackage{algorithmic}

% PACKAGES FOR REFERENCES & BIBLIOGRAPHY
\usepackage[colorlinks=true,linkcolor=black,anchorcolor=black,citecolor=black,filecolor=black,menucolor=black,runcolor=black,urlcolor=black]{hyperref} % Adds clickable links at references
\usepackage{cleveref}
% I CHANGED BELOW TWO LINES
\bibliographystyle{unsrtnat}
\usepackage[numbers,sort&compress]{natbib}

% OTHER PACKAGES
\usepackage{pdfpages} % To include a pdf file
\usepackage{afterpage}
\usepackage{lipsum} % DUMMY PACKAGE
\usepackage{fancyhdr} % For the headers
\fancyhf{}

% Input of configuration file. Do not change config.tex file unless you really know what you are doing. 
\input{Configuration_Files/config}

%----------------------------------------------------------------------------
%	NEW COMMANDS DEFINED
%----------------------------------------------------------------------------

% EXAMPLES OF NEW COMMANDS
\newcommand{\bea}{\begin{eqnarray}} % Shortcut for equation arrays
\newcommand{\eea}{\end{eqnarray}}
\newcommand{\e}[1]{\times 10^{#1}}  % Powers of 10 notation

%----------------------------------------------------------------------------
%	ADD YOUR PACKAGES (be careful of package interaction)
%----------------------------------------------------------------------------

%----------------------------------------------------------------------------
%	ADD YOUR DEFINITIONS AND COMMANDS (be careful of existing commands)
%----------------------------------------------------------------------------

%----------------------------------------------------------------------------
%	BEGIN OF YOUR DOCUMENT
%----------------------------------------------------------------------------

\begin{document}

\fancypagestyle{plain}{%
\fancyhf{} % Clear all header and footer fields
\fancyhead[RO,RE]{\thepage} %RO=right odd, RE=right even
\renewcommand{\headrulewidth}{0pt}
\renewcommand{\footrulewidth}{0pt}}

%----------------------------------------------------------------------------
%	TITLE PAGE
%----------------------------------------------------------------------------

\pagestyle{empty} % No page numbers
\frontmatter % Use roman page numbering style (i, ii, iii, iv...) for the preamble pages

\puttitle{
	title=Title Related to Microservices, % Title of the thesis
	name=Ömer Esas, % Author Name and Surname
	course=Computer Science and Engineering - Ingegneria Informatica, % Study Programme (in Italian)
	ID  = 917254,  % Student ID number (numero di matricola)
	advisor= Prof. Elisabetta Di Nitto, % Supervisor name
	coadvisor={Name Surname, Name Surname}, % Co-Supervisor name, remove this line if there is none
	academicyear={2021-22},  % Academic Year
} % These info will be put into your Title page 

%----------------------------------------------------------------------------
%	PREAMBLE PAGES: ABSTRACT (inglese e italiano), EXECUTIVE SUMMARY
%----------------------------------------------------------------------------
\startpreamble
\setcounter{page}{1} % Set page counter to 1

% ABSTRACT IN ENGLISH
\chapter*{Abstract} 
Here goes the Abstract in English of your thesis followed by a list of keywords.
\\
\\
\textbf{Keywords:} here, the keywords, of your thesis % Keywords

% ABSTRACT IN ITALIAN
\chapter*{Abstract in lingua italiana}
Qui va l'Abstract in lingua italiana della tesi seguito dalla lista di parole chiave.
\\
\\
\textbf{Parole chiave:} qui, vanno, le parole chiave, della tesi % Keywords (italian)

%----------------------------------------------------------------------------
%	LIST OF CONTENTS/FIGURES/TABLES/SYMBOLS
%----------------------------------------------------------------------------

% TABLE OF CONTENTS
\thispagestyle{empty}
\tableofcontents % Table of contents 
\thispagestyle{empty}
\cleardoublepage

%-------------------------------------------------------------------------
%	THESIS MAIN TEXT
%-------------------------------------------------------------------------
% In the main text of your thesis you can write the chapters in two different ways:
%
%(1) As presented in this template you can write:
%    \chapter{Title of the chapter}
%    *body of the chapter*
%
%(2) You can write your chapter in a separated .tex file and then include it in the main file with the following command:
%    \chapter{Title of the chapter}
%    \input{chapter_file.tex}
%
% Especially for long thesis, we recommend you the second option.

\addtocontents{toc}{\vspace{2em}} % Add a gap in the Contents, for aesthetics
\mainmatter % Begin numeric (1,2,3...) page numbering

% --------------------------------------------------------------------------
% NUMBERED CHAPTERS % Regular chapters following
% --------------------------------------------------------------------------
\chapter{Introduction}

\chapter{State of the Art}
\label{ch:art}%

\section{Microservice Architecture}
\label{sec:ms_arch}

Although the microservice architecture style has already been a de-facto standard for some large tech companies, and is being embraced by numerous firms in the industry, because of the novelty of the architecture, not all developers and architects in the tech industry and researchers in the academia are aware of what it means and which paradigms it advertises.
Microservice architecture is, not in the least meaning of the word, vastly different from the traditional way of building a web application, namely the monolithic architecture.
Hence, it is a valuable effort to define microservice architecture, what it is about and describe features and trends from which this rather unorthodox architecture emerged.
\\
Most importantly, microservice architecture is, as the name suggests, a software architecture.
There are numerous and slightly different definitions based on the particular discipline of software engineering for what a software architecture is.
However, a very simple yet powerful definition is, a (software) architecture is a representation of significant design decisions that shape a system, where significant is measured by the cost of change \cite{booch}. 
In the case for microservice architecture, the most signification design decision is, splitting the system into small, autonomous services that work together.
Focusing on each element of this design decision will bring about more clarity about the architecture.
\\
First, the microservice architecture divide the system into parts, as other architectural styles do, based on various point of views of the system.
Single Responsibility Principle, one of the famous SOLID principles of software engineering, promotes the idea that every module, class or a function in a computer program should have responsibility over a single part of that program's functionality, and it should encapsulate that part \cite{srp}.
The microservice architecture takes that idea to the extreme and encourages developing independent microservices that tackles just one business functionality.
Unlike a monolithic application, the system is not layered as database, back-end and front-end, or more generally, data, logic and UX layers, but consists of microservices that are created around business capabilities, as displayed in Figure~\ref{fig:monovsmicro}.

\begin{figure}[H]
    \centering
    \includegraphics[width=0.75\textwidth]{myImages/monolithic-vs-microservices.png}
    \caption{Monolithic vs Microservice Architecture}
    \label{fig:monovsmicro}
\end{figure}

Second, the microservice architecture advocates for those services to be small.
It is not easy and in some cases inaccurate (e.g, in terms of LOC) to give an estimate of the magnitude of a service, however, a rule of thumb to keep in mind is, microservices should be small enough and not smaller \cite{newman}.
Each service should focus on one business functionality and do it well.
\\
Third, and the last major aspect that defines the microservice architecture is autonomy.
Each service in the microservice architecture is a separate entity, even to the degree that they are mostly designed, developed and deployed by separate teams.
Each team has staff that can together carry out full range of skills required for development, such as database, UX and project management.
\\
At this point, in order to summarize the mentioned major aspects of the microservice architecture and make the architecture more concrete by adding a bit more detail about the implementation, it is a good opportunity to take a look at the definition of the microservice architecture given by an influential software engineer in the field.
According to M. Fowler, the microservice architecture is, "in short, an approach to developing a single application as a suite of small services, each running in its own process and communicating with lightweight mechanisms, often an HTTP resource API.
These services are built around business capabilities and independently deployable by fully automated deployment machinery.
There is a bare minimum of centralized management of these services, which may be written in different programming languages and use different data storage technologies." \cite{microdef}.
In the next section, each characteristic of the microservice architecture is explained in more detail.

\subsection{General Characteristics}
\label{subsec:chars}

\begin{itemize}
    \item Each microservice is developed by a small, cross-functional team.
    The team decides which programming language(s) and technology stack to choose to implement the microservice, and has their own CI/CD tools for testing, release and deployment.
    Each microservice is considered not just a project, but a product, and the development teams are responsible also for the deployment and production process of their microservice, in the Amazon's notion of "you build it, you run it" \cite{youbuild}.
    
    \item Each microservice is a light-weight component that is independently deployable. In case of a change in a particular library, systems that have multiple libraries in a single process like a monolithic architecture has to redeploy entire application. Instead, in a same scenario, having multiple services facilitates redeploying only the changed service. Moreover, this kind of ease in deployment enables the system to be more fault-tolerant and scalable in a more dynamic way, as illustrated Figure~\ref{fig:scalability}.
    
    \begin{figure}[H]
    \centering
    \includegraphics[width=0.75\textwidth]{myImages/scalable.png}
    \caption{Scalability in Monolithic vs Microservice Apps}
    \label{fig:scalability}
\end{figure}

    \item Microservices communicate with each other by means of network calls, using well-defined APIs, and simple protocols like REST over HTTP. While some other architectures incorporate smart (and heavy-weight) messaging mechanisms, such as Enterprise Service Buses (ESB) that can do routing, transformation, choreography and some business logic, the microservice architecture opt for simple communication infrastructure that can do basic routing of messages. In short, they have smart endpoints and dumb pipes.

    \item Each microservice is a loosely-coupled business unit, that is responsible for a single part of the business capability.
    Each model of a microservice is designed on a Bounded Context, which is a part of Domain Driven Design technique \cite{boundedcontext}. Conceptual model of the real world entities are decentralized, meaning that the representation (name) and modeling (attributes) of same real world entities are distinct. Figure~\ref{fig:micromodel} illustrates an example bounded context design and highlights the different representation of the same entity in different microservices.
    
    \begin{figure}[H]
    \centering
    \subfloat[Same Concept as Different Model Entities in Different Microservices\label{fig:micromodel1}]{
        \includegraphics[scale=0.4]{myImages/micromodel1.png}
    }
    \quad
    \subfloat[Decomposing Traditional Data Models\label{fig:micromodel2}]{
        \includegraphics[scale=0.45]{myImages/micromodel2.png}
    }
    \caption{A Microservices Design Using Bound Context Model per Microservice}
    \label{fig:micromodel}
\end{figure}

    \item Just like the decentralized modeling, the persistence layer of the whole application is decentralized, in other words, each microservice and associated team is responsible for managing their own data. The team decides on which kind of database (SQL, NoSQL, graph, columnar, etc) they make use of, taking into consideration their own models and needs.

\end{itemize}

\subsection{Differences from Service Oriented Architecture}
\label{subsec:diff}

The profound idea of microservice architecture, which proposes splitting a system into loosely-coupled, reusable, specialized components is not new.
In the late 90's, Service Oriented Architecture (SOA) emerged as an enterprise-wide approach to software development of components that takes advantage of reusable software components, or services.
Similar to microservice architecture, each service is designed to execute business functions.
\\
Although the two architectures look quite identical at the first glance, they take different different stands on the solutions of common problems in software architecture and therefore there are substantial differences between the two.
Listing the distinctions under three categories will help explain the difference.

\begin{itemize}
    \item Scope: SOA in general relates to enterprise-wide service exposure, while the microservice architecture has an application scope.
    The services are designed using common standards across development teams, aiming at re-usability and sharing of components, resources and data. On the other hand, microservices architecture embraces more relaxed governance approach, giving development teams more freedom of choice.
    Foregoing potential re-usability of code and data, microservice architecture prefers de-coupling of teams and services.
    
    \item Granularity: Having "re-usability across enterprise-wide system" in mind results in services that are fewer in number and larger in size in SOA. Each service typically handles more business functionality than microservices do. As for the persistence, SOA has a single data storage layer which is shared by all services, while each microservice has its own persistence mechanism, if needed for its specific business functionality. Although this results in data duplication in microservice architectures, it enables each microservice to be independent business unit in general \cite{soa_granularity}.
    Moreover, with respect to fine-grained microservices, coarse-grained services in SOA causes time-consuming deployment and less scalability.
    
    \item Communication: SOA makes use of ESB concept, which can handle, in addition to the communication between services using multiple protocols (RESTful API, SOAP, AMQP, MSMQ), management and configuration of services and even some business logic if needed \cite{soa_comm}.
    Having multiple capabilities like these can solve difficult integration problems in large scale systems, however, can possess the danger of single point of failure.
    In addition, the services across the enterprise frequently make synchronous calls, which can lead to latency issues and impact performance.
    To keep things simple, within an application scope, the microservice architecture prefers less elaborate and straightforward  messaging protocols such as HTTP, REST and Thrift.
    To provide communication and data synchronization across microservices, asynchronous communication models like event sourcing and pub/sub model are preferred.
\end{itemize}

\section{Design Patterns and Anti-Patterns in Microservices}
\label{sec:patterns}

Since its introduction by Netflix and discussions at workshops and software architecture conferences, the microservices architecture has gained quite a lot of popularity.
As the architecture is adopted more and more as time goes, legacy systems have been migrated and new projects have been developed utilizing the microservice architecture.
By sharing the experience after successful projects, similar to the evolution of design patterns in other paradigms, reusable solutions to commonly occurring problems have been identified and consequently design patterns in the microservice architecture showed up.
On the flip side, there has also been sub-optimal solutions during this period, resulting from several factors, some of which might be lack of experience, misunderstanding of the microservice architecture or just old habits from SOA.
In the same manner as design pattern, the anti-patterns of the microservice architecture has been identified by researcher and experienced engineers.
In the next two sections, the design patterns and anti-patterns are explored.

\subsection{Design Patterns}
\label{subsec:designpattern}

\subsubsection{API Gateway}
\label{subsubsec:api_gateway}

API gateway acts as a single entry point for all clients as well as an edge service for exposing microservices to the outside world as managed APIs.
It sounds like a reverse proxy, but also has additional responsibilities like simple load-balancing, authentication, authorization, failure handling, auditing, protocol translations, and routing. An API Gateway should always be a highly-available and performant component, since it is the entry point to the entire system, as illustrated in Figure~\ref{fig:api_gateway}.

\begin{figure}[H]
    \centering
    \includegraphics[width=0.75\textwidth]{myImages/api_gateway.png}
    \caption{An example of an API Gateway pattern}
    \label{fig:api_gateway}
\end{figure}

The most common duties of an API gateway include:

\begin{itemize}
    \item Gateway Aggregation: Aggregate multiple client requests (usually HTTP requests) targeting multiple internal microservices into a single client request, reducing chattiness and latency between consumers and services.
    
    \item Gateway Offloading: Enable individual microservices to offload their shared service functionality to the API gateway level.
    Such cross-cutting functionalities include authentication, authorization, service discovery, fault tolerance mechanisms, QoS, load balancing, logging, analytics etc.
    
    \item Gateway Routing (layer 7 routing, usually HTTP requests): Route requests to the endpoints of internal microservices using a single endpoint, so that consumers don’t need to manage many separate endpoints.
\end{itemize}

Developers can choose from implementing their own API gateway, using an existing API gateway solution such as Kong or Express-Gateway, or in case of cloud deployment, choose from products such as Google Apigee, AWS API Gateway or Azure API Gateway.

\subsubsection{Service Mesh with Sidecar}
\label{subsubsec:service_mesh}

A service mesh is a configurable , low-latency infrastructure layer aimed to handle high volume of network-based inter-process communication among  application infrastructure services through APIs \cite{service_mesh}.
Service mesh pattern is in general implemented with sidecar proxy in each service instance, that handles inter-process communication, monitoring and many other concerns.
Some aspects provided by this helper infrastructure include resiliency (fault tolerance, load balancing), service discovery, routing, observability, security, access control, communication protocol support and alike.
\\
The service mesh pattern is divided into two parts, namely, the control part and the data part, commonly referred as the control plane and the data plane. The control plane generates routing tables and deploy routing configuration to the proxies in the data plane. The actual forwarding of the network traffic is done by the proxies in the data plane, and for this reason, the data plane is also said to be the forwarding plane. Figure~\ref{fig:service_mesh} shows the diagram of an application with service mesh pattern, with the distinction of the control and data planes.

\begin{figure}[H]
    \centering
    \includegraphics[width=0.75\textwidth]{myImages/service_mesh.png}
    \caption{An Application Architecture utilizing Service Mesh with Sidecar Proxy }
    \label{fig:service_mesh}
\end{figure}

Some of the advantages of making use of a service mesh are:

\begin{itemize}
    \item Logic Decoupling: Decoupling of network communications from microservice business logic code allows developers to focus on the business capabilities.
    
    \item Routing: Primitive routing capabilities, but no routing logic related to the business functionality of the service.
    
    \item Resiliency for inter-service communications: Circuit-breaking, retries and timeouts, fault injection, fault handling, load balancing and fail-over. 
    
    \item Service Discover: Discovery of service endpoints through a dedicated service registry.
    
    \item Observability: Metrics, monitoring, distributed logging, distributed tracing.
    
    \item Security: Transport level security (TLS) and key management.
    
    \item Access Control: Simple blacklist and whitelist based access control.
    
    \item Deployment: Native support for containers, Docker and Kubernetes. Inter-service communication protocols: HTTP1.x, HTTP2, gRPC.
\end{itemize}

Implementations of the service mesh pattern include products such as Istio, Linkerd and Consul Connect.

\subsubsection{Backends For Frontends}
\label{subsubsec:bff}

Instead of using one common backend service for multiple clients, there are separate deployments of the same service with different configurations or implementations that can meet different UI requirements of different clients.
Each one is provides an API for its client.
Because each backend is specific to one interface, it can be optimized for that interface.
Each interface team has autonomy to control their own backend and doesn't rely on a centralized backend development team.

\subsubsection{Asynchronous Messaging}
\label{subsubsec:async_msg}

The distributed nature of microservices requires messaging mechanisms, ideally in a loosely-coupled manner.
The synchronous messaging results in tight run-time coupling, that is, both the client and the service need to be available during the whole messaging period.
To solve these issues and improve scalability, asynchronous messaging mechanisms are widely used in microservices architecture.
Solutions typically include light-weight event buses and message brokers.
Although an extra layer adds complexity, event buses and message brokers decrease run-time coupling by buffering messages, in other words, allowing the recipient to process messages when it becomes available.
Moreover, topics and content filtering can be used to create subsets of messages, delivered only to the interested parties.
With the help of built-in mechanisms of message brokers, different asynchronous messaging styles such as request/response, notification and publish/subscribe can be achieved.
Figure~\ref{fig:rabbitmq} illustrates an example diagram that includes RabbitMQ as a message broker, providing the publish/subscribe messaging manner.

\begin{figure}[H]
    \centering
    \includegraphics[width=0.75\textwidth]{myImages/rabbitmq.png}
    \caption{RabbitMQ Message Broker with Pub/Sub Mechanism}
    \label{fig:rabbitmq}
\end{figure}

Implementations of message brokers include RabbitMQ and Apache Kafka. The Key-Value store Redis can also be used as a message broker. In addition, cloud providers offers message brokers and event buses with different capabilities, such as AWS SNS, AWS SQS, AWS Eventbridge, Azure Event Bus, Google Cloud Tasks and Google Cloud Pub/Sub.


\subsection{Anti-Patterns}
\label{subsec:antipattern}

\chapter{Adopted Methodology}
\label{ch:method}%

\section{Classification of Patterns}
\label{sec:class_method}

\section{Selection of Open Source Projects}
\label{sec:select_method}

\section{Detection of Patterns}
\label{sec:detection_method}

\chapter{Results}
\label{ch:results}%

\section{Resulting Classification of Patterns}
\label{sec:class_result}

\section{Selected Open Source Projects}
\label{sec:select_result}

\section{Detected Patterns in Selected Projects}
\label{sec:detection_result}

\section{Discussion of Findings}
\label{sec:discussion}

\chapter{Conclusion}
\label{ch:conclusion}%

%-------------------------------------------------------------------------
%	BIBLIOGRAPHY
%-------------------------------------------------------------------------

\addtocontents{toc}{\vspace{2em}} % Add a gap in the Contents, for aesthetics
\bibliography{myThesisBib}

%-------------------------------------------------------------------------
%	APPENDICES
%-------------------------------------------------------------------------

\cleardoublepage
\addtocontents{toc}{\vspace{2em}} % Add a gap in the Contents, for aesthetics
\appendix
\chapter{Appendix A}
If you need to include an appendix to support the research in your thesis, you can place it at the end of the manuscript.
An appendix contains supplementary material (figures, tables, data, codes, mathematical proofs, surveys, \dots)
which supplement the main results contained in the previous chapters.

% LIST OF FIGURES
\listoffigures

% LIST OF TABLES
\listoftables

% LIST OF ABBREVIATIONS
% Write out the List of Symbols in this page
\chapter*{List of Abbreviations}
\begin{table}[H]
    \centering
    \begin{tabular}{ll}
        \textbf{Abbreviation} & \textbf{Description} \\\hline\\[-9px]
        AMQP & Advanced Messaging Queuing Protocol \\[2px]
        API & Application Programming Interface \\[2px]
        BC & Bounded Context \\[2px]
        CI/CD & Continuous Integration / Continuous Delivery \\[2px]
        DDD & Domain Driven Design \\[2px]
        ESB & Enterprise Service Bus \\[2px]
        HTTP & Hypertext Transfer Protocol \\[2px]
        LOC & Lines of Code \\[2px]
        MSMQ & Microsoft Messaging Queuing \\[2px]
        NoSQL & Not-Only-SQL, to refer to different kinds of non-relational databases \\[2px]
        QoS & Quality of Service \\[2px]
        REST & Representational State Transfer \\[2px]
        RESTful API & an API that adheres to REST principles \\[2px]
        SOA & Service Oriented Architecture \\[2px]
        SOAP & Simple Object Access Protocol \\[2px]
        SQL & Structured Query Language \\[2px]
        TLS & Transport Layer Security \\[2px]
        UI & User Interface \\[2px]
        UX & User Experience \\[2px]
        
    \end{tabular}
\end{table}

% ACKNOWLEDGEMENTS
\chapter*{Acknowledgements}
Here you might want to acknowledge someone.

\cleardoublepage

\end{document}
