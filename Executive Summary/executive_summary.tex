% A LaTeX template for EXECUTIVE SUMMARY of the MSc Thesis submissions to 
% Politecnico di Milano (PoliMi) - School of Industrial and Information Engineering
%
% S. Bonetti, A. Gruttadauria, G. Mescolini, A. Zingaro
% e-mail: template-tesi-ingind@polimi.it
%
% Last Revision: October 2021
%
% Copyright 2021 Politecnico di Milano, Italy. NC-BY

\documentclass[11pt,a4paper,twocolumn]{article}

%------------------------------------------------------------------------------
%	REQUIRED PACKAGES AND  CONFIGURATIONS
%------------------------------------------------------------------------------
% PACKAGES FOR TITLES
\usepackage{titlesec}
\usepackage{color}

% PACKAGES FOR LANGUAGE AND FONT
\usepackage[utf8]{inputenc}
\usepackage[english]{babel}
\usepackage[T1]{fontenc} % Font encoding

% PACKAGES FOR IMAGES
\usepackage{graphicx}
\graphicspath{{Images/}} % Path for images' folder
\usepackage{eso-pic} % For the background picture on the title page
\usepackage{subfig} % Numbered and caption subfigures using \subfloat
\usepackage{caption} % Coloured captions
\usepackage{transparent}

% STANDARD MATH PACKAGES
\usepackage{amsmath}
\usepackage{amsthm}
\usepackage{bm}
\usepackage[overload]{empheq}  % For braced-style systems of equations

% PACKAGES FOR TABLES
\usepackage{tabularx}
\usepackage{longtable} % tables that can span several pages
\usepackage{colortbl}

% PACKAGES FOR ALGORITHMS (PSEUDO-CODE)
\usepackage{algorithm}
\usepackage{algorithmic}

% PACKAGES FOR REFERENCES & BIBLIOGRAPHY
\usepackage[colorlinks=true,linkcolor=black,anchorcolor=black,citecolor=black,filecolor=black,menucolor=black,runcolor=black,urlcolor=black]{hyperref} % Adds clickable links at references
\usepackage{cleveref}
% I CHANGED BELOW TWO LINES
\usepackage[numbers,sort&compress]{natbib}
\bibliographystyle{unsrtnat}

% PACKAGES FOR THE APPENDIX
\usepackage{appendix}

% PACKAGES FOR ITEMIZE & ENUMERATES 
\usepackage{enumitem}

% OTHER PACKAGES
\usepackage{amsthm,thmtools,xcolor} % Coloured "Theorem"
\usepackage{comment} % Comment part of code
\usepackage{fancyhdr} % Fancy headers and footers
\usepackage{lipsum} % Insert dummy text
\usepackage{tcolorbox} % Create coloured boxes (e.g. the one for the key-words)
\usepackage{stfloats} % Correct position of the tables

%-------------------------------------------------------------------------
%	NEW COMMANDS DEFINED
%-------------------------------------------------------------------------
% EXAMPLES OF NEW COMMANDS -> here you see how to define new commands
\newcommand{\bea}{\begin{eqnarray}} % Shortcut for equation arrays
\newcommand{\eea}{\end{eqnarray}}
\newcommand{\e}[1]{\times 10^{#1}}  % Powers of 10 notation
\newcommand{\mathbbm}[1]{\text{\usefont{U}{bbm}{m}{n}#1}} % From mathbbm.sty
\newcommand{\pdev}[2]{\frac{\partial#1}{\partial#2}}
% NB: you can also override some existing commands with the keyword \renewcommand

%----------------------------------------------------------------------------
%	ADD YOUR PACKAGES (be careful of package interaction)
%----------------------------------------------------------------------------
\usepackage{pifont}
\newcommand{\cmark}{\ding{51}}%
\newcommand{\xmark}{\ding{55}}%

\usepackage{graphicx}
\newcommand{\Csharp}{%
  {\settoheight{\dimen0}{C}C\kern-.05em \resizebox{!}{\dimen0}{\raisebox{\depth}{\# }}}}
%----------------------------------------------------------------------------
%	ADD YOUR DEFINITIONS AND COMMANDS (be careful of existing commands)
%----------------------------------------------------------------------------


% Do not change Configuration_files/config.tex file unless you really know what you are doing. 
% This file ends the configuration procedures (e.g. customizing commands, definition of new commands)
\input{Configuration_files/config}

% Insert here the info that will be displayed into your Title page 
% -> title of your work
\renewcommand{\title}{Design Patterns and Anti-Patterns in Microservices Architecture: A Classification Proposal and Study on Open Source Projects}
% -> author name and surname
\renewcommand{\author}{Ömer Esas}
% -> MSc course
\newcommand{\course}{Computer Science and Engineering - Ingegneria Informatica}
% -> advisor name and surname
\newcommand{\advisor}{Prof. Elisabetta Di Nitto}
% IF AND ONLY IF you need to modify the co-supervisors you also have to modify the file Configuration_files/title_page.tex (ONLY where it is marked)
% \newcommand{\firstcoadvisor}{Name Surname} % insert if any otherwise comment
%\newcommand{\secondcoadvisor}{Name Surname} % insert if any otherwise comment
% -> academic year
\newcommand{\YEAR}{2021-2022}

%-------------------------------------------------------------------------
%	BEGIN OF YOUR DOCUMENT
%-------------------------------------------------------------------------
\begin{document}

%-----------------------------------------------------------------------------
% TITLE PAGE
%-----------------------------------------------------------------------------
% Do not change Configuration_files/TitlePage.tex (Modify it IF AND ONLY IF you need to add or delete the Co-advisors)
% This file creates the Title Page of the document
\input{Configuration_files/title_page}

%%%%%%%%%%%%%%%%%%%%%%%%%%%%%%
%%     THESIS MAIN TEXT     %%
%%%%%%%%%%%%%%%%%%%%%%%%%%%%%%

%-----------------------------------------------------------------------------
% INTRODUCTION
%-----------------------------------------------------------------------------
\section{Introduction}
\label{sec:introduction}

As tech giants such as Amazon, Netflix and Uber adopted microservices architecture successfully in the past decade, there has been greater interest from the academia and the industry towards the microservices architecture and its principles.
The attempts to implement applications using microservices architecture led to the emergence of desirable and undesirable ways of solving common problems faced in distributed applications, namely the design patterns and anti-patterns.
In this study, we aim to observe the way the design patterns and anti-patterns are classified in the literature, check if there exists a common way of classification and propose our own taxonomy in case there is no consensus in the literature.
Then, we select ten open source microservice projects and manually inspect source code to detect the design patterns and anti-patterns of microservices architecture, in order to observe the correlation between the "theory" in the literature and practical cases to some extent.
\\
To the best of our knowledge, there are two studies in the literature that are similar to our work.
The researchers of the first similar study \cite{8719492} inspect a set of thirty open source microservice projects using automated tools that check the dependency files and verify the use of the pattern by checking the documentation of the utilized framework.
The researchers of the second similar study \cite{10.1145/3424771.3424812} manually check sixty seven projects to detect anti-patterns, which they discover as a result of their systematic literature review for microservices anti-patterns.
Our study differs from the two studies in considering not only design patterns or anti-patterns, but both of these good and bad practices, in addition to using only manual inspection on projects that use many different technologies.

%-----------------------------------------------------------------------------
% STATE OF THE ART
%-----------------------------------------------------------------------------

\section{State of the Art}
\label{sec:State of the Art}

%Before presenting the research methodology and the results, we briefly explain the microservices %architecture, and the design patterns and anti-patterns around this structure.

\subsection{Design Patterns}
\label{subsec:patterns}

Being a distributed system with unusual domain design, microservices architecture comes with some problems regarding various aspects in software engineering.
Thanks to the research and past experiences of developers in microservices area, it has been observed that there are in fact a number of techniques that can be embraced to solve the common issues faced in microservice-based systems.
These methods are called "design patterns" of the microservices architecture and can help software architects and developers design and implement scalable, fault-tolerant and performant microservices applications.
A few examples from the set of design patterns of microservices architecture considered in this study are shortly described below, which are obtained after reviewing sources such as microservices patterns book \cite{richardson_book}, Microsoft Cloud patterns website \cite{microsoft_docs} and papers that contain classifications of design patterns \cite{TaibiD2018APfM}\cite{KARABEYAKSAKALLI2021111014}\cite{valdivia}\cite{10.1145/3241403.3241429}\cite{8719492}\cite{9105640}.\\

\begin{itemize}
    \item API Gateway: Using a intermediary service between frontend and microservices to route requests to specific microservices depending on the request.
    \item Service Registry and Discovery: Having an "address-book" in terms of host and port numbers for microservices to let them locate each other and make API calls.
    \item Asynchronous Messaging: Utilizing a helper service to receive, temporarily store (buffer) and distribute messages between microservices.
    \item Command Query Responsibility Segregation (CQRS): Having multiple microservice instances that solely focus on either "read" or "write" tasks of the application.
    \item Event Sourcing: Embracing event-driven communication, storing events in a event store to persist data in a temporal way.
    \item Service Instance per Container: Creating and deploying Docker containers per microservice.
    \item Health Check API: Implementing an additional API endpoint in microservices to check the status of instances periodically
    \item Distributed Tracing: Emitting metadata from microservices about recent API calls, to be received by a central service for combining metadata to observe the "journey" of an API call as a chain of different microservices.
\end{itemize}

\subsection{Anti-Patterns}
\label{subsec:anti-patterns}

On the other hand, when utilized without a proper understanding of its principles, microservices architecture might result in numerous issues, from poor performance and recurring service failures to inefficient development and maintenance efforts.
Similarly, it has been identified by researchers that there are a number of prevalent approaches that are adopted by developers, which might cause issues in a microservice-based system.
These common bad practices are called "anti-patterns" of microservices architecture, and architects and developers are advised to avoid implementing these wrong or sub-optimal techniques and keep in mind the key concepts of this architecture.
We briefly describe below some of the anti-patterns considered during this study, which are obtained from two papers \cite{9522227}\cite{10.1145/3424771.3424812}.\\

\begin{itemize}
    \item Wrong Cut: Dividing application into technical layers or services, instead of business capabilities and bounded contexts.
    \item Hardcoded Endpoints: Instead of using service discovery, hardcoding the address (host and port number) of a microservice.
    \item No API Gateway: Directly calling backend microservices from the client without intermediary service.
    \item Shared Persistence: Utilization of the same database instance by multiple microservices.
    \item No CI/CD Tools: Not making use of CI/CD tools in shared repositories in a distributed development context.
    \item No API Versioning: Not having version prefixes inside the URLs of API definitions.
    \item Local Logging: Not aggregating logs of microservices in a central service, absence of log aggregator pattern.
\end{itemize}

\section{Research Methodology}
\label{sec:research_method}

As the goal of this study, we aimed to observe the classification regarding microservices design patterns and anti-patterns in the literature, present one in case there is no consensus, and check the presence of these patterns and anti-patterns in popular open source projects.
Specifically, we established two research questions:
\begin{itemize}
    \item \textbf{RQ1}: Is there a consistent categorization or classification of design patterns and anti-patterns of microservices architecture in the academia?
    If not, what could be an alternative way to classify these design patterns and anti-patterns?
    
    \item \textbf{RQ2}: Which of these design patterns and anti-patterns exist in popular open source microservices applications?
\end{itemize}

To answer first research question, the following steps have been adopted.

\paragraph{Querying digital libraries}
In order to find out whether there exists a consistent classification of microservice design patterns and anti-patterns, a literature review on digital libraries such as IEEE Explore, ACM Digital Library, Springer, Scopus, and Google Scholar has been carried out.
The keywords used in the search queries included "microservice pattern", "microservice pattern classification", "microservice anti-pattern" and "microservice anti-pattern classification".

\paragraph{Applying snowballing}
After initial review of the studies found, few more studies have been added through snowballing technique, and from this extended set, the studies that do not contain a classification or grouping of patterns and anti-patterns have been eliminated.

\paragraph{Consulting systematic mapping studies}
To be able to propose a classification of patterns and anti-patterns on a sound basis, various perspectives through which microservices architecture can be examined needed to be identified.
For this reason, the systematic mapping studies found in the literature review process have been consulted.

\paragraph{Developing classification proposal}
Considering the classifications found in the literature and the findings resulted from papers that conducted systematic mapping studies, a classification proposal has been developed.\\

As for the second research question, the following steps have been carried out.

\paragraph{Querying GitHub}
Open source repository hosting service GitHub is queried with he keyword pattern "microservice OR micro-service". 
To find the most popular projects, the search result is sorted using "most stars" option.

\paragraph{Selecting projects}
Ten microservice repositories that have the most number of stars are selected.
While doing so, the repositories that contain libraries, frameworks or tool-kits have been eliminated, in addition to the ones that contain "microservice" keyword but rather implemented using other architectures.

\paragraph{Excluding saga pattern and shared libraries anti-pattern}
Because of the technological heterogeneity among the set of projects, two design patterns, namely "saga" and "shared libraries" which require a competent understanding of the programming language being used and a thorough comprehension of the business logic are excluded from the list of patterns and anti-patterns to be identified.

\paragraph{Detecting patterns and anti-patterns}
We manually inspected the repositories and tried to find any information on the repository page that indicates the use of a particular pattern or anti-pattern. We examined the source code, deployment files such as "docker-compose.yaml" and dependency files such as "pom.xml" when available, and then tried to verify the use of a pattern by consulting to the reference document of a library or framework at hand.

\section{Results}
\label{sec:result}

\subsection{Classification of Patterns and Anti-Patterns}
\label{subsec:rq1}

As a result of the research process related to the classification of microservices patterns and anti-patterns in the literature, we found out that there are three types of classifications for design patterns and three types of categorizations for the anti-patterns, which result from a set of nine studies and one book chapter \cite{TaibiD2018APfM}\cite{KARABEYAKSAKALLI2021111014}\cite{valdivia}\cite{10.1145/3241403.3241429}\cite{8719492}\cite{9105640}\cite{Taibi2020}\cite{10.1145/3424771.3424812}\cite{8712355}\cite{9522227}. 
Therefore, we concluded that there is no consensus in the literature regarding the classification of patterns and anti-patterns of microservices architecture.
While developing the proposed classification, we noticed that there are categories in the reviewed papers such as "design", "communication", "back-end", "coordination", "data" and "architectural". We argued that one pattern or anti-pattern that is placed in one of those categories in fact also affects or is affected from the ones placed in other similar categories.
As an example, while the "hardcoded endpoints" anti-pattern is placed under "implementation" category in \cite{10.1145/3424771.3424812}, the bad practice results from a lack of service discovery mechanism in the architecture of the application.
We therefore combined the patterns and anti-pattern that we think are about the major design decision, i.e., architecture of the application into "architectural" category.
For the remaining ones, we realised that they reflect the processes of deployment and monitoring stages of a software development lifecycle and they are about the decisions that affect the most those stages.
We then presented our taxonomy proposal by suggesting to use "architectural", "deployment" and "monitoring \& reliability" categories as shown in Table~\ref{table:pattern_table_1}, Table~\ref{table:pattern_table_2} and Table~\ref{table:pattern_table_3} respectively.

\begin{table}[H]
\centering 
    \begin{tabular}{ 
  | >{\centering\arraybackslash} m{9em} 
  | >{\centering\arraybackslash} m{9em} | }
    \hline
    \rowcolor{bluepoli!40}
    \textbf{Architectural Patterns} & \textbf{Architectural Anti-Patterns}\T\B \\
    \hline \hline
    API Gateway & Wrong Cut\T\B\\
    \hline
    \rowcolor{bluepoli!10}
    Service Mesh with Sidecar & Nano Microservice\T\B\\
    \hline
    Service Registry \& Discovery & Mega Microservice\T\B\\
    \hline
    \rowcolor{bluepoli!10}
    Backends for Frontends & ESB Usage\T\B\\
    \hline
    Asynchronous Messaging & Shared Libraries\T\B\\
    \hline
    \rowcolor{bluepoli!10}
    Database per Service & Hardcoded Endpoints \T\B\\
    \hline
    Saga & No API Gateway \T\B\\
    \hline
    \rowcolor{bluepoli!10}
    API Composition & Shared Persistence \T\B\\
    \hline
    CQRS & \T\B\\
    \hline
    \rowcolor{bluepoli!10}
    Event Sourcing & \T\B\\
    \hline
    \end{tabular}
    \\[10pt]
    \caption{Architectural patterns and anti-patterns of microservices architecture}
    \label{table:pattern_table_1}
\end{table}

\begin{table}[H]
\centering 
    \begin{tabular}{ 
  | >{\centering\arraybackslash} m{9em} 
  | >{\centering\arraybackslash} m{9em} | }
    \hline
    \rowcolor{bluepoli!40}
    \textbf{Deployment Patterns} & \textbf{Deployment Anti-Patterns}\T\B \\
    \hline \hline
    Service Instance per Container & No CI/CD\T\B\\
    \hline
    \rowcolor{bluepoli!10}
    Service Instance per VM & Multiple Service Instances per Host\T\B\\
    \hline
    Serverless & No API Versioning \T\B\\
    \hline
    \end{tabular}
    \\[10pt]
    \caption{Deployment patterns and anti-patterns of microservices architecture}
    \label{table:pattern_table_2}
\end{table}

\begin{table}[H]
\centering 
    \begin{tabular}{ 
  | >{\centering\arraybackslash} m{9em} 
  | >{\centering\arraybackslash} m{9em} | }
    \hline
    \rowcolor{bluepoli!40}
    \textbf{Monitoring \& Reliability Patterns} & \textbf{Monitoring \& Reliability Anti-Patterns}\T\B \\
    \hline \hline
    Health Check & No Health Check\T\B\\
    \hline
    \rowcolor{bluepoli!10}
    Distributed Tracing & Local Logging\T\B\\
    \hline
    Log Aggregator & \T\B\\
    \hline
    \rowcolor{bluepoli!10}
    Circuit Breaker & \T\B\\
    \hline
    \end{tabular}
    \\[10pt]
    \caption{Monitoring \& reliability patterns and anti-patterns of microservices architecture}
    \label{table:pattern_table_3}
\end{table}

\subsection{Patterns and Anti-Patterns in Microservice Projects}
\label{subsec:rq2}

The set of examined projects is presented in Table~\ref{table:project_list}, with their names and number of GitHub stars.
\begin{table}[H]
\centering 
    \begin{tabular}{ 
  | >{\centering\arraybackslash} m{15.3em} 
  | >{\centering\arraybackslash} m{3.5em} | }
    \hline
    \rowcolor{bluepoli!40}
    \textbf{Repository Name} & \textbf{\#Stars}\T\B \\
    \hline \hline
    dotnet-architecture/eShopOnContainers & 20.3k\T\B\\
    \hline
    \rowcolor{bluepoli!10}
    GoogleCloudPlatform/microservices-demo & 12k\T\B\\
    \hline
    sqshq/piggymetrics & 11.5k\T\B\\
    \hline
    \rowcolor{bluepoli!10}
    cer/event-sourcing-examples & 2.9k\T\B\\
    \hline
    microservices-patterns/FTGO-application & 2.4k\T\B\\
    \hline
    \rowcolor{bluepoli!10}
    vietnam-devs/coolstore-microservices & 2k\T\B\\
    \hline
    Crizstian/cinema-microservice & 1.6k\T\B\\
    \hline
    \rowcolor{bluepoli!10}
    asc-lab/dotnetcore-microservices-poc & 1.5k\T\B\\
    \hline
    elgris/microservice-app-example & 1.4k\T\B\\
    \hline
    \rowcolor{bluepoli!10}
    aspnetrun/run-aspnetcore-microservices & 1.1k\T\B\\
    \hline
    \end{tabular}
    \\[10pt]
    \caption{List of examined projects}
    \label{table:project_list}
\end{table}

Next, starting from the first project on the list, we examined repository page and source code to detect the aforementioned patterns and anti-patterns.
To report the results, we first gave a short summary of the application domain and used technologies, indicated the presence of each pattern and anti-pattern in a table and then explained each pattern and anti-pattern by referring to a particular file in the repository when available.

% For exemplary purposes, the presence of design patterns and anti-patterns discovered in the "eShopOnContainers" application is indicated in Table~\ref{table:eshop_1} and Table~\ref{table:eshop_2} respectively.

% \begin{table}[H]
% \centering 
%     \begin{tabular}{ 
%   | >{\centering\arraybackslash} m{15.5em} 
%   | >{\centering\arraybackslash} m{2.2em} | }
%     \hline
%     \rowcolor{bluepoli!40}
%     \textbf{Design Pattern} & \cmark \textbackslash – \T\B \\
%     \hline \hline
%     API Gateway & \cmark \T\B\\
%     \hline
%     \rowcolor{bluepoli!10}
%     Service Mesh with Sidecar & \cmark \T\B \\
%     \hline
%     Service Registry \& Discovery & \cmark \T\B \\
%     \hline
%     \rowcolor{bluepoli!10}
%     Backends for Frontends & \cmark \T\B \\
%     \hline
%     Asynchronous Messaging & \cmark \T\B \\
%     \hline
%     \rowcolor{bluepoli!10}
%     Database per Service & – \T\B \\
%     \hline
%     API Composition & \cmark \T\B \\
%     \hline
%     \rowcolor{bluepoli!10}
%     CQRS & \cmark \T\B \\
%     \hline
%     Event Sourcing & – \T\B \\
%     \hline
%     \rowcolor{bluepoli!10}
%     Service Instance per VM & – \T\B \\
%     \hline
%     Service Instance per Container & \cmark \T\B \\
%     \hline
%     \rowcolor{bluepoli!10}
%     Serverless & – \T\B \\
%     \hline
%     Health Check & \cmark \T\B \\
%     \hline
%     \rowcolor{bluepoli!10}
%     Distributed Tracing & – \T\B \\
%     \hline
%     Log Aggregator & \cmark\T\B \\
%     \hline
%     \rowcolor{bluepoli!10}
%     Circuit Breaker & – \T\B \\
%     \hline
%     \end{tabular}
%     \\[10pt]
%     \caption{Presence of microservice design patterns in eShopOnContainers application}
%     \label{table:eshop_1}
% \end{table}

% \begin{table}[H]
% \centering 
%     \begin{tabular}{ 
%   | >{\centering\arraybackslash} m{15.5em} 
%   | >{\centering\arraybackslash} m{2.2em} | }
%     \hline
%     \rowcolor{bluepoli!40}
%     \textbf{Anti-Pattern} & \cmark \textbackslash – \T\B \\
%     \hline \hline
%     Wrong Cut & – \T\B\\
%     \hline
%     \rowcolor{bluepoli!10}
%     Nano Microservice & – \T\B \\
%     \hline
%     Mega Microservice & – \T\B \\
%     \hline
%     \rowcolor{bluepoli!10}
%     ESB Usage & – \T\B \\
%     \hline
%     Hardcoded Endpoints & – \T\B \\
%     \hline
%     \rowcolor{bluepoli!10}
%     No API Gateway & – \T\B \\
%     \hline
%     Shared Persistence & \cmark \T\B \\
%     \hline
%     \rowcolor{bluepoli!10}
%     No CI/CD & – \T\B \\
%     \hline
%     Multiple Service Instances per Host & – \T\B \\
%     \hline
%     \rowcolor{bluepoli!10}
%     No API Versioning & – \T\B \\
%     \hline
%     No Health Check & – \T\B \\
%     \hline
%     \rowcolor{bluepoli!10}
%     Local Logging & – \T\B \\
%     \hline
%     \end{tabular}
%     \\[10pt]
%     \caption{Presence of microservice anti-patterns in eShopOnContainers application}
%     \label{table:eshop_2}
% \end{table}

\subsection{Discussion of Findings}
\label{subsec:discussion}

As a result of the detection process to find out which patterns and anti-patterns exists in prominent open source projects, we also think that it is a valuable effort to take a look at the total number of patterns and anti-patterns found in the ten projects examined.

\begin{table}[H]
\centering 
    \begin{tabular}{ 
  | >{\centering\arraybackslash} m{15.5em} 
  | >{\centering\arraybackslash} m{2em} | }
    \hline
    \rowcolor{bluepoli!40}
    \textbf{Design Pattern} & \textbf{\#} \T\B \\
    \hline \hline
    API Gateway & 10 \T\B\\
    \hline
    \rowcolor{bluepoli!10}
    Service Mesh with Sidecar & 3 \T\B \\
    \hline
    Service Registry \& Discovery & 8 \T\B \\
    \hline
    \rowcolor{bluepoli!10}
    Backends for Frontends & 1\T\B \\
    \hline
    Asynchronous Messaging & 7 \T\B \\
    \hline
    \rowcolor{bluepoli!10}
    Database per Service & 2  \T\B \\
    \hline
    API Composition & 2 \T\B \\
    \hline
    \rowcolor{bluepoli!10}
    CQRS & 5 \T\B \\
    \hline
    Event Sourcing & 2 \T\B \\
    \hline
    \rowcolor{bluepoli!10}
    Service Instance per VM & – \T\B \\
    \hline
    Service Instance per Container & 10 \T\B \\
    \hline
    \rowcolor{bluepoli!10}
    Serverless & – \T\B \\
    \hline
    Health Check & 6 \T\B \\
    \hline
    \rowcolor{bluepoli!10}
    Distributed Tracing & 5 \T\B \\
    \hline
    Log Aggregator & 3 \T\B \\
    \hline
    \rowcolor{bluepoli!10}
    Circuit Breaker & 2 \T\B \\
    \hline
    \end{tabular}
    \\[10pt]
    \caption{Total number of design patterns in examined projects}
    \label{table:total_number_1}
\end{table}

As shown in Table~\ref{table:total_number_1}, we see that the API gateway, service registry and discovery and asynchronous messaging are the most widely used architectural design patterns among the ten projects.
Next, we observe that using containers is the preferred approach by far when compared to virtual machine images and serverless deployment when it comes to deploying a microservice application.
As for the anti-patterns, we notice in Table~\ref{table:total_number_2} that the most frequent anti-pattern among the ten projects is the no API versioning anti-pattern, possibly because the examined applications are not actual microservice products that are maintained by a number of different development teams, it might be deemed not necessary by developers of examined repositories to make use of API versioning practice.
\begin{table}[H]
\centering 
    \begin{tabular}{ 
  | >{\centering\arraybackslash} m{15.5em} 
  | >{\centering\arraybackslash} m{2em} | }
    \hline
    \rowcolor{bluepoli!40}
    \textbf{Anti-Pattern} & \textbf{\#} \T\B \\
    \hline \hline
    Wrong Cut & – \T\B\\
    \hline
    \rowcolor{bluepoli!10}
    Nano Microservice & – \T\B \\
    \hline
    Mega Microservice & – \T\B \\
    \hline
    \rowcolor{bluepoli!10}
    ESB Usage & – \T\B \\
    \hline
    Hardcoded Endpoints & 5 \T\B \\
    \hline
    \rowcolor{bluepoli!10}
    No API Gateway & – \T\B \\
    \hline
    Shared Persistence & 6 \T\B \\
    \hline
    \rowcolor{bluepoli!10}
    No CI/CD & 5 \T\B \\
    \hline
    Multiple Service Instances per Host & – \T\B \\
    \hline
    \rowcolor{bluepoli!10}
    No API Versioning & 8 \T\B \\
    \hline
    No Health Check & 4 \T\B \\
    \hline
    \rowcolor{bluepoli!10}
    Local Logging & 7 \T\B \\
    \hline
    \end{tabular}
    \\[10pt]
    \caption{Total number of anti-patterns in examined projects}
    \label{table:total_number_2}
\end{table}


Coming to the architectural anti-patterns, we observe that the design principles of microservice architectures are well digested by the practitioners.
The microservices are designed around business capabilities in a balanced way and the principle of "smart endpoints, dumb pipes" is put into practice in those designs.
%-----------------------------------------------------------------------------
% CONCLUSION
%-----------------------------------------------------------------------------
\section{Conclusions}

With this study, we investigated the literature about classifications regarding microservice patterns and anti-patterns, and observed that there are a number of different categorizations.
By taking into account the way these studies categorize the patterns and anti-patterns and by constructing our own argumentation, we presented our taxonomy proposal by suggesting to utilize "architectural", "deployment" and "monitoring \& reliability" categories, in order to provide a simple and valid structure in terms of classification for patterns and anti-patterns.
Furthermore, we manually inspected ten open source microservice projects to see if those patterns and anti-patterns are actually present in implemented microservice architectures.
By inspecting the total number of patterns and anti-patterns, we observed the imbalance in frequency of patterns and anti-patterns and discussed the probable reasons.
\\
Regarding the possible future work, by inspecting more open source projects, the ability to generalise the result might be increased, and focusing on projects that make use of the same framework and the same technologies, a precise and thorough understanding of the implemented business logic could be achieved, enabling more observations about patterns and anti-patterns related to the logic of the application.

%---------------------------------------------------------------------------
%  BIBLIOGRAPHY
%---------------------------------------------------------------------------
% Remember to insert here only the essential bibliography of your work
\bibliography{bibliography.bib} % automatically inserted and ordered with this command 

\end{document}